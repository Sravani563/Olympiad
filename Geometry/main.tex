\documentclass[journal,12pt,onecolumn]{IEEEtran}
%
\usepackage{setspace}
\usepackage{gensymb}
%\doublespacing
\singlespacing

%\usepackage{graphicx}
%\usepackage{amssymb}
%\usepackage{relsize}
\usepackage[cmex10]{amsmath}
%\usepackage{amsthm}
%\interdisplaylinepenalty=2500
%\savesymbol{iint}
%\usepackage{txfonts}
%\restoresymbol{TXF}{iint}
%\usepackage{wasysym}
\usepackage{amsthm}
%\usepackage{iithtlc}
\usepackage{mathrsfs}
\usepackage{txfonts}
\usepackage{stfloats}
\usepackage{bm}
\usepackage{cite}
\usepackage{cases}
\usepackage{subfig}
%\usepackage{xtab}
\usepackage{longtable}
\usepackage{multirow}
%\usepackage{algorithm}
%\usepackage{algpseudocode}
\usepackage{enumitem}
\usepackage{mathtools}
\usepackage{tikz}
\usepackage{circuitikz}
\usepackage{verbatim}
%\usepackage{tfrupee}
\usepackage[breaklinks=true]{hyperref}
%\usepackage{stmaryrd}
\usepackage{tkz-euclide} % loads  TikZ and tkz-base
\usetkzobj{all}
\usepackage{listings}
    \usepackage{color}                                            %%
    \usepackage{array}                                            %%
    \usepackage{longtable}                                        %%
    \usepackage{calc}                                             %%
    \usepackage{multirow}                                         %%
    \usepackage{hhline}                                           %%
    \usepackage{ifthen}                                           %%
  %optionally (for landscape tables embedded in another document): %%
    \usepackage{lscape}     
\usepackage{multicol}
\usepackage{chngcntr}
%\usepackage{enumerate}

%\usepackage{wasysym}
%\newcounter{MYtempeqncnt}
\DeclareMathOperator*{\Res}{Res}
%\renewcommand{\baselinestretch}{2}
\renewcommand\thesection{\arabic{section}}
\renewcommand\thesubsection{\thesection.\arabic{subsection}}
\renewcommand\thesubsubsection{\thesubsection.\arabic{subsubsection}}

\renewcommand\thesectiondis{\arabic{section}}
\renewcommand\thesubsectiondis{\thesectiondis.\arabic{subsection}}
\renewcommand\thesubsubsectiondis{\thesubsectiondis.\arabic{subsubsection}}

% correct bad hyphenation here
\hyphenation{op-tical net-works semi-conduc-tor}
\def\inputGnumericTable{}                                 %%

\lstset{
%language=C,
frame=single, 
breaklines=true,
columns=fullflexible
}
%\lstset{
%language=tex,
%frame=single, 
%breaklines=true
%}

\begin{document}
%


\newtheorem{theorem}{Theorem}[section]
\newtheorem{problem}{Problem}
\newtheorem{proposition}{Proposition}[section]
\newtheorem{lemma}{Lemma}[section]
\newtheorem{corollary}[theorem]{Corollary}
\newtheorem{example}{Example}[section]
\newtheorem{definition}[problem]{Definition}
%\newtheorem{thm}{Theorem}[section] 
%\newtheorem{defn}[thm]{Definition}
%\newtheorem{algorithm}{Algorithm}[section]
%\newtheorem{cor}{Corollary}
\newcommand{\BEQA}{\begin{eqnarray}}
\newcommand{\EEQA}{\end{eqnarray}}
\newcommand{\define}{\stackrel{\triangle}{=}}

\bibliographystyle{IEEEtran}
%\bibliographystyle{ieeetr}


\providecommand{\mbf}{\mathbf}
\providecommand{\pr}[1]{\ensuremath{\Pr\left(#1\right)}}
\providecommand{\qfunc}[1]{\ensuremath{Q\left(#1\right)}}
\providecommand{\sbrak}[1]{\ensuremath{{}\left[#1\right]}}
\providecommand{\lsbrak}[1]{\ensuremath{{}\left[#1\right.}}
\providecommand{\rsbrak}[1]{\ensuremath{{}\left.#1\right]}}
\providecommand{\brak}[1]{\ensuremath{\left(#1\right)}}
\providecommand{\lbrak}[1]{\ensuremath{\left(#1\right.}}
\providecommand{\rbrak}[1]{\ensuremath{\left.#1\right)}}
\providecommand{\cbrak}[1]{\ensuremath{\left\{#1\right\}}}
\providecommand{\lcbrak}[1]{\ensuremath{\left\{#1\right.}}
\providecommand{\rcbrak}[1]{\ensuremath{\left.#1\right\}}}
\theoremstyle{remark}
\newtheorem{rem}{Remark}
\newcommand{\sgn}{\mathop{\mathrm{sgn}}}
\providecommand{\abs}[1]{\left\vert#1\right\vert}
\providecommand{\res}[1]{\Res\displaylimits_{#1}} 
\providecommand{\norm}[1]{\left\lVert#1\right\rVert}
%\providecommand{\norm}[1]{\lVert#1\rVert}
\providecommand{\mtx}[1]{\mathbf{#1}}
\providecommand{\mean}[1]{E\left[ #1 \right]}
\providecommand{\fourier}{\overset{\mathcal{F}}{ \rightleftharpoons}}
%\providecommand{\hilbert}{\overset{\mathcal{H}}{ \rightleftharpoons}}
\providecommand{\system}{\overset{\mathcal{H}}{ \longleftrightarrow}}
	%\newcommand{\solution}[2]{\textbf{Solution:}{#1}}
\newcommand{\solution}{\noindent \textbf{Solution: }}
\newcommand{\cosec}{\,\text{cosec}\,}
\providecommand{\dec}[2]{\ensuremath{\overset{#1}{\underset{#2}{\gtrless}}}}
\newcommand{\myvec}[1]{\ensuremath{\begin{pmatrix}#1\end{pmatrix}}}
\newcommand{\mydet}[1]{\ensuremath{\begin{vmatrix}#1\end{vmatrix}}}
%\numberwithin{equation}{section}
\numberwithin{equation}{subsection}
%\numberwithin{problem}{section}
%\numberwithin{definition}{section}
\makeatletter
\@addtoreset{figure}{problem}
\makeatother

\let\StandardTheFigure\thefigure
\let\vec\mathbf
%\renewcommand{\thefigure}{\theproblem.\arabic{figure}}
\renewcommand{\thefigure}{\theproblem}
%\setlist[enumerate,1]{before=\renewcommand\theequation{\theenumi.\arabic{equation}}
%\counterwithin{equation}{enumi}


%\renewcommand{\theequation}{\arabic{subsection}.\arabic{equation}}

\def\putbox#1#2#3{\makebox[0in][l]{\makebox[#1][l]{}\raisebox{\baselineskip}[0in][0in]{\raisebox{#2}[0in][0in]{#3}}}}
     \def\rightbox#1{\makebox[0in][r]{#1}}
     \def\centbox#1{\makebox[0in]{#1}}
     \def\topbox#1{\raisebox{-\baselineskip}[0in][0in]{#1}}
     \def\midbox#1{\raisebox{-0.5\baselineskip}[0in][0in]{#1}}

\vspace{3cm}

\title{
%	\logo{
Geometry: Maths Olympiad
%	}
}
\author{ G V V Sharma$^{*}$% <-this % stops a space
	\thanks{*The author is with the Department
		of Electrical Engineering, Indian Institute of Technology, Hyderabad
		502285 India e-mail:  gadepall@iith.ac.in. All content in this manual is released under GNU GPL.  Free and open source.}
	
}	
%\title{
%	\logo{Matrix Analysis through Octave}{\begin{center}\includegraphics[scale=.24]{tlc}\end{center}}{}{HAMDSP}
%}


% paper titles
% can use linebreaks \\ within to get better formatting as desired
%\title{Matrix Analysis through Octave}
%
%
% author names and IEEE memberships
% note positions of commas and nonbreaking spaces ( ~ ) LaTeX will not break
% a structure at a ~ so this keeps an author's name from being broken across
% two lines.
% use \thanks{} to gain access to the first footnote area
% a separate \thanks must be used for each paragraph as LaTeX2e's \thanks
% was not built to handle multiple paragraphs
%

%\author{<-this % stops a space
%\thanks{}}
%}
% note the % following the last \IEEEmembership and also \thanks - 
% these prevent an unwanted space from occurring between the last author name
% and the end of the author line. i.e., if you had this:
% 
% \author{....lastname \thanks{...} \thanks{...} }
%                     ^------------^------------^----Do not want these spaces!
%
% a space would be appended to the last name and could cause every name on that
% line to be shifted left slightly. This is one of those "LaTeX things". For
% instance, "\textbf{A} \textbf{B}" will typeset as "A B" not "AB". To get
% "AB" then you have to do: "\textbf{A}\textbf{B}"
% \thanks is no different in this regard, so shield the last } of each \thanks
% that ends a line with a % and do not let a space in before the next \thanks.
% Spaces after \IEEEmembership other than the last one are OK (and needed) as
% you are supposed to have spaces between the names. For what it is worth,
% this is a minor point as most people would not even notice if the said evil
% space somehow managed to creep in.



% The paper headers
%\markboth{Journal of \LaTeX\ Class Files,~Vol.~6, No.~1, January~2007}%
%{Shell \MakeLowercase{\textit{et al.}}: Bare Demo of IEEEtran.cls for Journals}
% The only time the second header will appear i/year/1963s for the odd numbered pages
% after the title page when using the twoside option.
% 
% *** Note that you probably will NOT want to include the author's ***
% *** name in the headers of peer review papers.                   ***
% You can use \ifCLASSOPTIONpeerreview for conditional compilation here if
% you desire.




% If you want to put a publisher's ID mark on the page you can do it like
% this:
%\IEEEpubid{0000--0000/00\$00.00~\copyright~2007 IEEE}
% Remember, if you use this you must call \IEEEpubidadjcol in the second
% column for its text to clear the IEEEpubid ma/year/1963rk.



% make the title area
\maketitle



%\tableofcontents

\bigskip

\renewcommand{\thefigure}{\theenumi}
\renewcommand{\thetable}{\theenumi}
%\renewcommand{\theequation}{\theenumi}

%\begin{abstract}
%%\boldmath
%In this letter, an algorithm for evaluating the exact analytical bit error rate  (BER)  for the piecewise linear (PL) combiner for  multiple relays is presented. Previous results were available only for upto three relays. The algorithm is unique in the sense that  the actual mathematical expressions, that are prohibitively large, need not be explicitly obtained. The diversity gain due to multiple relays is shown through plots of the analytical BER, well supported by simulations. 
%
%\end{abstract}
% IEEEtran.cls defaults to using nonbold math in the Abstract.
% This preserves the distinction between vectors and scalars. However,
% if the journal you are submitting to favors bold math in the abstract,
% then you can use LaTeX's standard command \boldmath ast the very start
% of the abstract to achieve this. Many IEEE journals frown on math
% in the abstract anyway.

% Note that keywords are not normally used for peerreview papers.
%\begin{IEEEkeywords}
%Cooperative diversity, decode and forward, piecewise linear
%\end{IEEEkeywords}



% For peer review papers, you can put extra information on the cover
% page as needed:
% \ifCLASSOPTIONpeerreview
% \begin{center} \bfseries EDICS Category: 3-BBND \end{center}
% \fi
%
% For peerreview papers, this IEEEtran command inserts a page break and
% creates the second title. It will be ignored for othesr modes.
%\IEEEpeerreviewmaketitle


%Download python codes using 
%\begin{lstlisting}
%svn co https://github.com/gadepall/school/trunk/ncert/computation/codes
%\end{lstlisting}

\renewcommand{\theequation}{\theenumi}
\begin{enumerate}[label=\arabic*.,ref=\theenumi]
%\begin{enumerate}[label=\arabic*.,ref=\thesubsection.\theenumi]
\numberwithin{equation}{enumi}

\item Find the smallest natural number n which has the following properties:
\begin{enumerate}
\item Its decimal representation has 6 as the last digit.
\item If the last digit 6 is erased and placed in front of the remaining digits,
the resulting number is four times as large as the original number n.
\end{enumerate}


\item In a mathematical competition, in which 6 problems were posed to the participants, every two of these problems were solved by more than $\frac{2}{5}$of the contestants. Moreover, no contestant solved all the 6 problems.Show that there are at least 2 contestants who solved exactly 5 problems each.

\item Let ABC be a triangle with incentre I. A point P in the interior of the
triangle satisfies
\begin{align*}
\angle PBA + \angle PCA = \angle PBC + \angle PCB.
\end{align*}
Show that AP $\geq$ AI, and that equality holds if and only if P = I.

\item Let P be a regular 2006-gon. A diagonal of P is called good if its endpoints divide the boundary of P into two parts, each composed of an odd number of sides of P .The sides of P are also called good.
Suppose P has been dissected into triangles by 2003 diagonals, no two of which have a common point in the interior of P . Find the maximum number of isosceles triangles having two good sides that could appear in such a configuration.

\item Assign to each side b of a convex polygon P the maximum area of a triangle that has b as a side and is contained in P . Show that the sum of the areas assigned to the sides of P is at least twice the area of P.

\item Real numbers $a_1, a_2,......., a_n$ are given. For each i ($1 \leq i \leq n$) define
\begin{align*}
d_i = max\{a_j : 1 \leq j \leq i\} - min\{a_j : i \leq j \leq  n\}
\end{align*}
and let
\begin{align*} 
d = max\{d_i : 1\leq i \leq n\}.
\end{align*}
\begin{enumerate}
\item Prove that, for any real numbers $ x_1 \leq x_2 \leq$....$\leq x_n$,
      max$\{|x_i - a_i| : 1\leq i \leq n\} \geq \frac{d}{2}$.
\item Show that there are real numbers $x_1 \leq x_2 \leq $........$\leq x_n $such that equality holds in.
\end{enumerate} 

\item Let a and b be positive integers. Show that if 4ab-1 divides $(4a^{2}-1)^{2}$,
then a = b.

\item Let n be a positive integer. Consider
\begin{align*}
S = \{(x, y, z) : x, y, z\in  \{0, 1,......., n\}, x + y + z > 0\}
\end{align*}
as a set of $(n+1)^3 - 1$ points in three-dimensional space. Determine the smallest possible number of planes, the union of which contains S but does not include (0, 0, 0).


\item Find all functions f : (0, $\infty$) $\rightarrow$ (0,$\infty$) (so, f is a function from the positive real numbers to the positive real numbers) such that
\begin{align*}
\frac{(f(w))^2 + (f(x))^2}{f(y^2) + f(z^2)}=\frac{w^2 + x^2}{y^2 + z^2}
\end{align*}
for all positive real numbers w, x, y, z, satisfying wx = yz.


\item Let $a_1, a_2,......., a_n $be distinct positive integers and let M be a set of n - 1 positive integers not containing $s = a_1 + a_2 +........+ a_n$. A grasshopper is to jump along the real axis, starting at the point 0 and making n jumps to the right with lengths $a_1, a_2,......., a_n$ in some order. Prove that the order can be chosen in such a way that the grasshopper never lands on any point in M.


\item Let $a_1, a_2, a_3,.......$ be a sequence of positive real numbers. Suppose that for some positive integer s, we have 
\begin{align*}
a_n = max\{a_k + a_{n-k} | 1 \leq k \leq n - 1\}
\end{align*}
for all $n > s$. Prove that there exist positive integers l and N, with $l\leq s$ and such that  $a_n = a_l+a_{n - l}$ for all $n \leq N$.

\item Let ABC be an acute triangle with circumcircle $\Gamma$. Let l be a tangent line to $\Gamma$, and let $l_a,l_b$ and $l_c$ be the lines obtained by reflecting l in the lines BC, CA and AB, respectively. Show that the circumcircle of the triangle determined by the lines $l_a,l_b$ and $l_c$  is tangent to the circle $\Gamma$.

\item Find all positive integers n for which there exist non-negative integers $a_1, a_2,......,a_n$ such that
\begin{align*}
\frac{1}{2^{a_1}} + \frac{1}{2^{a_2}}+....+\frac{1}{2^{a_n}}
= \frac{1}{3^{a_1}} + \frac{2}{3^{a_2}}+....+\frac{n}{3^{a_n}} = 1
\end{align*}





\item Let the excircle of triangle ABC opposite the vertex A be tangent to the side BC at the point $A_1$. Define the points $B_1$ on CA and $C_1$ on AB analogously, using the excircles opposite B and C, respectively. Suppose that the circumcentre of triangle $A_1B_1C_1$ lies on the circumcircle of triangle ABC. Prove that triangle ABC is right-angled.The excircle of triangle ABC opposite the vertex A is the circle that is tangent to the line segment BC, to the ray AB beyond B, and to the ray AC beyond C. The excircles opposite B and C are similarly defined.

\item Let ABC be an acute-angled triangle with orthocentre H, and let W be a point on the side BC, lying strictly between B and C. The points M and N are the feet of the altitudes from B and C, respectively. Denote by $\omega_1$ the circumcircle of BWN, and let X be the point on $\omega_1$ such that WX is a diameter of $\omega_1$. Analogously, denote by $\omega_2$ the circumcircle of CWM, and let Y be the point on $\omega_2$ such that WY is a diameter of $\omega_2$. Prove that X, Y and H are collinear.

\item Let $a_0 < a_1 < a_2 < $ · · · be an infinite sequence of positive integers. Prove that there exists a unique integer n $\leq$ 1 such that
\begin{align*}
a_n < \frac{a_0 + a_1 +. . . .+ a_n}{n} \leq a_{n+1}
\end{align*}

\item Let ABC be an acute triangle with AB $>$ AC.Let $\Gamma$ be its circumcircle,H its orthocentre,and F the foot of the altitude from A.Let M be the midpoint of BC.Let Q be the point on $\Gamma$ suchn that $\angle$HQA=90,and let K be the point on $\Gamma$ such that $\angle$HKQ=90.Assume that the point A,B,C,K and Q are all different,and lie on $\Gamma$ in this order.
Prove that the circumcircle of triangle KQH and FKM are tangent to each other.

\item Triangle ABC has cicumcircle $\Omega$ and circumcenter O.A circle $\Gamma$ with centre A intersects the segment BC at point D and E,such that B,D,E and C are all different and lie on the line BC in this order .Let F and G be the points of intersection of $\Gamma$ and $\Omega$,such that A,F,B,C and G lie on $\Omega$ in this order.Let K be the second point of intersection of the circumcircle of the triangle BDF and segment AB.Let L be the second point of intersection of circumcircle of triangle CGE and the segment CA.
Suppose that lines FK and Gl are different and intersects at the point X.Prove that X lies on the line AO.


\item A set of positive integers is called fragrant if it contains at least two elements and each of its elements has a prime factor in common with at least one of the other elements.Let P(n) = $n^2 + n + 1$
what is the least possible value of the positive integer b such that there exists a non-negative integer a for which the set\\
$\{P(a+1),P(a+2),...,P(a+b)\}$\\
is fragrant?
 

\item For each integer $a_0 > 1$, define the sequence $a_0, a_1, a_2,$. . . .  by:\\
$a_{n+1}$ = \[ \begin{cases} 
      \sqrt{a_n} if \sqrt{a_n} \enspace is \enspace an \enspace integer, \\
      a_{n+3}\enspace otherwise \\
       
   \end{cases}
\]
for all each n $\geq$ 0\\
Determine all values of $a_0$ for which there is a number A such that $a_n$ = A for infinitely many values of n.
\item An ordered pair (x, y) of integers is a primitive point if the greatest common divisor
of x and y is 1. Given a finite set S of primitive points, prove that there exist a positive integer n and integers $a_0, a_1,$ . . . , $a_n$ such that, for each (x, y) in S, we have:
\begin{align*}
a_0 x^n + a_1 x^{n - 1}y + a_2 x^{n - 2}y^2 +. . . . . + a_{n-1}xy^{n-1} + a_n y^n = 1.
\end{align*}
 
\item Find all integers n $\geq$ 3 for which there exist real numbers $a_1, a_2,$ . . . , $a_n+2$, such that 
$a_{n + 1} = a_1$ and $a_{n + 2} = a_2$, and $a_ia_{i + 1} + 1 = a_{i + 2}$ for i = 1, 2, . . . , n.

\item Let $a_1,a_2$, . . . be an infinite sequence of positive integers. Suppose that there is an
integer N $>$ 1 such that, for each n $\geq$ N, the number
\begin{align*}
\frac{a_1}{a_2} + \frac{a_2}{a_3}+.....+\frac{a_{n-1}}{a_n}+\frac{a_n}{a_1}
\end{align*}
is an integer. Prove that there is a positive integer M such that$a_m = a_{m+1}$ for all m $\geq$ M.
 
\end{enumerate}

\end{document}


